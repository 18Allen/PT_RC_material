\documentclass[12pt]{article}

\usepackage[english]{babel}

\usepackage[letterpaper,top=2cm,bottom=2cm,left=3cm,right=3cm,marginparwidth=1.75cm]{geometry}

% Useful packages
\usepackage{amsmath}
\usepackage{amsfonts}
\usepackage{dsfont}
\usepackage{graphicx}
\usepackage[colorlinks=true, allcolors=blue]{hyperref}
%\usepackage[legalpaper, landscape, margin=1in]{geometry}
\usepackage{geometry}
\geometry{
a4paper,
left=25mm,
right=25mm,
top=15mm,
}
\usepackage{setspace}
\setstretch{1.25}
\usepackage{soul}

\begin{document}


\section*{Solution to HW 8}
\subsection*{Erreta}
Q5: \dots Are \st{X} and Y independent? \(\to \) \dots Are Z and Y independent?  
\subsection*{Heads up}
This homework is more or less about the change of variable for single/multi variate p.d.f.\\
Before you start, think about what tools you have at hand. You may use any that fit the problem. 
\begin{itemize}
   \item Change of Variable formula (CoV) \\
   (p.55~56 for single variable, p.60~61 for multi-variable) 
   This method is mention also in RC7 as well. Just remember that you need to put a measure-correction term $\left\vert \frac{\partial x}{\partial y} \right\vert $ (for single variable case) or \(\left\vert J_T \right\vert \) (for multivariable case) \\
   For single variable case, changing from $x$ to $y$ with transformation $T(y)$  
   \[
    g(y) = f(T(y)) \left\vert \frac{\partial x}{\partial y} \right\vert 
   \]  
    
   For 2 variable case (multi-variable is similar), changing a coordinate $(x,y)$ to \((u,v)\) with transformation $(x,y) = T(u,v)$, we can get the new p.d.f. $g(u,v)$ by the original p.d.f $f(x,y)$  by 
   \[
    g(u,v) = f(T(u,v)) \left| J_T \right|
   \]
   \item Cumulative distribution function (c.d.f.) 
   (p.54~55, \textbf{only for single variable case})
   As we know that for cont. r.v., the way to 'measure' the probability of an event is to integrate the p.d.f of that event. 
   Hence, you can think of this method as a version that you \textbf{integrate first, then derivate} to get new p.d.f. 
   We use (TBD) as examples.  
\end{itemize}
\subsection*{Q1}
\textbf{Sol.}\\ 
First, we have to determine the range of \(Y\). You can check that \(Y \in [1,\infty )\). \\  
Using the c.d.f. method, we have (note that there should be two cases, \(x \in [0,0.5]\) and \(x \in [0.5,1]\)  )
\[
   P(Y \leq  y) = P(\frac{1}{1+y} \leq X \leq 0.5)+P(0.5\leq X \leq \frac{y}{1+y})  = \frac{y-1}{1+y},\quad  y\in [1,\infty)
\]
Then, do derivative on both side we have
\[
   f_{Y} (y) = \frac{d}{dy}P(Y \leq y) = \begin{cases}
      \frac{2}{(1+y)^{2}}  , &\text{ if } y\in [1,\infty ) ;\\
     0 , &\text{ ow }  .
   \end{cases}
\]
\hspace{\textwidth}\(\square\) 
\subsection*{Q2}
\textbf{Sol.}\\ 
\begin{itemize}
    \item \textbf{\(V = \frac{X}{1-X}\)}\\
    Same methods as in Q1.  
    \item \textbf{\(W = X(1-X)\)}\\
    First, we have to determine the range of \(W\). Since \(X(1-X) = -(X-\frac{1}{2})^{2} +\frac{1}{4}\), we have \(W \in [0,\frac{1}{4}]\). 
    
    Using c.d.f. method, we can write

    \[
       P(W \leq w) = P(X(1-X) \leq w) = P(X \leq \frac{1-\sqrt{1-4w} }{2}) + P(X \leq \frac{1+\sqrt{1-4w} }{2}), w \in [0,1/4]
    \]  
    which equals to 
    \[
      P(W \leq w) = 1 - \sqrt{1-4w}, w\in [0,\frac{1}{4}]
    \] 
    Do derivate on both sides, we have 
    \[
      f_{W} (w) = \frac{d}{dw}P(W \leq w) = \begin{cases}
         \frac{2}{\sqrt{1-4w} }, &\text{ if }  w \in [0,\frac{1}{4}] \\ 
         0, &\text{ ow }.\\
      \end{cases}
    \]
    \hspace{\textwidth}\(\square \) 
\end{itemize}

\subsection*{Q3}
\textbf{Sol.}\\ 
This is an obvious example that use CoV method.\\
Since \(X\sim N(\mu ,\sigma ^{2})\), the p.d.f. of \(X\) is \(f_{X} (x) = \frac{1}{\sqrt{2\pi \sigma ^{2} } } \exp (-(x-\mu )^2 /(2\sigma ^{2} )\)\\
If we let \(U = \frac{X-\mu }{\sigma }\), the transformation \(x = T(u)\) we need is \(T(u) = \sigma (u) + \mu \). 
Before we proceed, it is important you remember \textbf{multiplying with scalar and linear shift will keep a Gaussian variable Gaussian, just with different parameter}. \\  
By using the description for single variable case we have 
\[
   f_{U} (u) = f_{X} (T(u)) \left\vert \frac{dT}{du} \right\vert = \frac{1}{\sqrt{2\pi } } \exp{\frac{-u^2}{2}} 
\]     
By the definition of Guassian r.v., \(U  = \frac{X - \mu }{\sigma } \sim N(0,1)\).  
\\ \hspace{\textwidth}\(\square\) \\
\(-X\) case is easy with \(T(u) = -u\).   
\subsection*{Q4}
\textbf{Sol.}\\ 
   This is an exercise of multi-variable integral. Do it yourself. 
\begin{itemize}
    \item The joint p.d.f. of \((X,Y,Z)\) \\
    You may assume that they distribute uniformly. 
    \item \(P(X^{2}  > YZ)\)
    \[
      \int_0^1 (x^2 - x^2\ln{x^2}) dx =  \frac{1}{3} + \frac{2}{9} = \frac{5}{9}
    \]
    \item \(P(X+Y < Z)\)
    \[
      \frac{1}{6}
    \]
    \item \(P(\max{(X,Y)} > Z)\)   
    \[
      \frac{2}{3}
    \]
\end{itemize}

\subsection*{Q5}
\textbf{Sol.}\\ 
\begin{itemize}
    \item The p.d.f. of \(Z = \min{(X,Y)}\) 
    Obviously, the range of \(Z\) is \([0,1]\).
    We use c.d.f. method here. 
    \[
      P(Z \leq z) = P(\min{(X,Y)} \leq z) = 1 - P(X > z \cap Y > z) = 1 - (1-z)^2, z \in [0,1]
    \]
    Hence, 
    \[
      f_Z(z) = P^\prime(Z \leq z) = 2(1-z), \quad z \in [0,1]
    \]
    \hspace{\textwidth} \(\square\) 
    \item Are \(Z\) and \(Y\) independent?\\ 
    It is unlikely right? A counterexample to consider is the events \(Z > 0.5\) and \(Y \leq 0.1\).  
\end{itemize}
\subsection*{Q6}
\textbf{Sol.} \\
Since \(X\) and \(Y\) are indep. , the joint p.d.f. \(f(x,y)\) is 
\[
   f(x,y) = f_X(x)f_{Y} (y) = \frac{1}{2\pi } e^{-\frac{x^{2} +y^{2}  }{2}}
\] 
\textbf{First way: Polar coordinate \& CoV} \\
Remember how you will solve a Gaussian integral \href{https://en.wikipedia.org/wiki/Gaussian_integral#By_polar_coordinates}{wiki}, 
We consider turning \((X,Y)\) to \((r,\theta )\) with \(T(r,\theta ) = (r\cos{\theta },r\sin{\theta} ), r \in \mathbb{R}^+, \theta \in (0,2\pi )\)  
Then by CoV method we have the new joint p.d.f. as 
\[
   h(r,\theta ) = f(T(r,\theta )) \left\vert \frac{\partial (x,y)}{\partial (r,\theta )} \right\vert  = r\frac{1}{2\pi } e^{-\frac{r^{2} }{2}}
\]
Then, the marginal p.d.f. of \(r\) is 
\[
   h_r(r) = \int_0^{2\pi }h(r,\theta ) d \theta  = r e^{-\frac{r^{2}}{2} }, r \in \mathbb{R}^+
\]
We are not done yet! \(r = \sqrt{X^2 + Y^{2} } \) but we want \(X^{2}  + Y^{2} \). Let \(u = r^2\), we can use CoV for single variable to acquire the new p.d.f. of \(u\) as 
\[
  f_U(u) = h_r(\sqrt{u}) \left\vert \frac{d r}{du} \right\vert  = \frac{1}{2}e^{-\frac{u}{2}}, \quad u \in \mathbb{R}^+.  
\]
In other word, \(U \sim \text{Exp}(\frac{1}{2})\). 
\\ \hspace{\textwidth}\(\square\)      
\\
\textbf{Another way: Using method in p.61}\\
You can also find the p.d.f. of \(X^2\) and \(Y^2\) first and multiplying to get the joint p.d.f. of \((X^2, Y^2)\). It can work because if \(X,Y\) are indep. then \(g(X), f(Y)\) is also indep.
Finally, use the method for \(X + Y\) introduced in p.61 . 


\subsection*{Q7}
\textbf{Sol.} \\
\begin{itemize}
    \item \(\min{(X,Y)} \sim \text{Exp}(\alpha +\beta )\) \\ 
    Use the same method of c.d.f. in Q5 but in this case \(P(X > z \cap Y > z)\) is different. 
    \item \(P(X < Y)\)\\  
    This is a direct computation. 
    \[
      P(X < Y) = \int_0^{\infty} \alpha e^{-\alpha  x}\int_x^{\infty}  \beta  e^{- \beta y} dydx = \frac{\alpha }{\beta + \alpha } 
    \]
\end{itemize}
\subsection*{Q8}
\textbf{Sol.} \\
Base case: \(n = 2\)
\[
   P(X_1 + X_2 \leq  s) = \int_0^s \int_0^{s-x_1}  dx_2 dx_1 = \frac{s^2}{2}   
\] 
Induction hypothesis: The statement is true up to \(K \in \mathbb{N}\). \\ 
Induction step: Consider \(n = K +1\)
\begin{equation*}
    \begin{aligned}
        & P((X_1 + \dots X_K) + X_{K+1} \leq s)
        \\ &= \int_0^s P( (X_1 + \dots X_K) \leq  s- x_{K+1} ) dx_{K+1} 
        \\ &= \int_0^s \frac{(s-x_{K+1})^K}{K!} dx_{K+1} 
        \\ &= \frac{s^{K+1}}{(K+1)!}
    \end{aligned}
\end{equation*} 
Hence, the hypothesis is also true for \(n = K +1\). 
By mathematical induction, the statement is true for all \(n\in \mathbb{N}\). 
\\ \hspace{\textwidth}\(\square\)   
\subsection*{Q9}
\textbf{Sol.} \\
\textbf{It will only work for gaussian random variables that covariance equals zero implies independent} (Prop. 14)\\  
Hence, the goal of this problem is to show that \(X+Y\) and \(X-Y\) are uncorrelated.\\
For multiple normal r.v. problem, their joint p.d.f. will be a good starting point. 
Let \(Z = (X,Y)\), the p.d.f of random vector \(Z\) is (p.73 or RC note Theorem 8.1.1) 
\[
   h(z) = \frac{1}{(2\pi) (\text{det}(\Sigma) )^{\frac{1}{2}}} e^{-\frac{1}{2} (z)^T \Sigma ^{-1} (z) }
\]  
where \(\Sigma  = \begin{bmatrix}
   1 &  0 \\
   0 &  1 \\
\end{bmatrix}\). 
Next step, finding the transformation matrix \(T\). 
\[
   \begin{bmatrix}
       X+Y \\
       X-Y \\
   \end{bmatrix} = \begin{bmatrix}
      1 &  1 \\
      1 &  -1 \\
   \end{bmatrix} \begin{bmatrix}
       X \\
       Y \\
   \end{bmatrix}, 
\]  
which means 
\[
   \begin{bmatrix}
       X \\
       Y \\
   \end{bmatrix} = \underbrace{\frac{1}{2} \begin{bmatrix}
      1 &  1 \\
      1 &  -1 \\
   \end{bmatrix}}_{T} \begin{bmatrix}
       X+Y \\
       X-Y \\
   \end{bmatrix}, 
\] 
Hence, given \(W = [X+Y,X-Y]^{\top} \), we have \(z = Tw\), so 
\[
   z^{\top} \Sigma^{-1} z = w^{\top} \underbrace{T^{\top} \Sigma^{-1} T}_{\Sigma_W^{-1}} w
\]
where 
\[
   \Sigma_{W}  = 4 \begin{bmatrix}
      1 &  0 \\
      0 &  1 \\
   \end{bmatrix}
\] 
By Prop. 13, it implies \(X+Y\) and \(X-Y\) are independent with \(\sigma ^{2} = 4 \) \\
\hspace{\textwidth}\(\square\) \\
\textbf{Note}\\
We stopped at \(\Sigma_{W} \), but if you want the p.d.f. of \(W\), using the CoV method you have 
\begin{equation*}
    \begin{aligned}
   h_W(w) &= h_Z(Tw) \left\vert \frac{\partial (X,Y)}{\partial (X+Y,X-Y)} \right\vert  
         \\ &= h_Z(Tw) |\text{det}(T)|
         \\ &=   \frac{1}{(2\pi) (\text{det}(\Sigma_W) )^{\frac{1}{2}}} e^{-\frac{1}{2} (w)^T \Sigma_W ^{-1} (z) }
    \end{aligned}
\end{equation*}
with \(\det{\Sigma_W} = 16\) 
 
\subsection*{Q10}
\textbf{Sol.} \\
\begin{itemize}
    \item The joint p.d.f. of \(V = X+Y\) and  \(W = \frac{X}{X+Y}\).
    \\ Again starting with the joint p.d.f. of \((X,Y)\), which is 
    \[
      f(x,y) = f_X(x)f_Y(y) = \lambda ^{2} e^{-\lambda (x+y)}
    \]  
    The first equality is due to the independence of \(X,Y\).  
    \\Since we have \((X,Y) = T(V,W) = (WV, V(1-W))\), the Jacobian is
    \[
      J_T = \begin{bmatrix}
         w &  v \\
         1-w & -v  \\
      \end{bmatrix}
    \] 
    Hence,  \(|\text{det}(J_T)| = v\).
    Using CoV method we may write the joint p.d.f. of \(V,W\) as 
    \[
      h(v,w) = f(T(v,w))\left\vert \text{det}(J_T) \right\vert = \lambda ^2 v e^{-\lambda (v)} , \quad (v,w) \in [0,\infty) \times [0,1]  
    \]   

    \item Prove that \(V\) and \(W\) are independent. 
    \\By the definition of independent for continuous r.v., we will show that \(h(z,w) = f_Z(z)f_W(w)\). 
    \\ 
     Therefore, we have to find \(f_Z(z)\) and \(f_W(w)\). \\ 
     \begin{itemize}
         \item \(f_Z(z)\):\\
         From \(h(z,w)\) we can get \(f_Z(z)\) by taking the marginal. 
         \[
            f_Z(z) = \int_0^1 h(v,w)dw = \lambda ^2 v e^{-\lambda v}, v\in [0, \infty)
         \]   
         \item \(f_W(w)\):\\  
         Taking the marginal for \(w\) we have \(f_{W}(w) = 1 , w \in [0,1]\). 
     \end{itemize}  
     Finally, since \(h(z,w)\) is the product of marignal p.d.f., \(V\) and \(W\) are independent.  
\end{itemize}
\subsection*{Q11}
\textbf{Sol.} \\
\begin{itemize}
    \item The joint p.d.f. of \((X^2,Y^2)\). \\ 
    Here since both \(V\) and \(W\) use the second moment, so it is convienient to 
    get the distribution of \(X^2\) and \(Y^2\) and use the "Another way" mentioned in Q6. 
    If \(A \) is an event in \(\mathbb{R}^+\), the probability of \(U = X^2 \in A\) is    
    \begin{equation*}
        \begin{aligned}
            P(U \leq u) &= P(-\sqrt{u} \leq  X\leq \sqrt{u} )
                     \\ &= 2P(0\leq X \leq \sqrt{u} )
        \end{aligned}
    \end{equation*}
    The last equation does not have a exact form, and we don't need that either. 
    Taking the derivative against \(u\) 
    \[
      f_U(u) = \frac{d}{du} 2 P(0 \leq X \leq \sqrt{u} ) = \frac{1}{\sqrt{2\pi u}} e^{-\frac{u}{2}}
    \]  
    On the other hand, since \(X,Y\) are independent \(X^2, Y^2\) are indepent, too.
   Hence, the joint p.d.f. of \((X^{2} , Y^{2} )\) is 
   \[
      f(x,y) = \frac{1}{2\pi \sqrt{xy} } e^{-\frac{1}{2}(x+y)}
   \]  
   We can finally return to the original problem. 
   \item The joint p.d.f. of \((V,W)\)\\ 
   \[
      \begin{bmatrix}
          X^2 \\
          Y^2 \\
      \end{bmatrix} = \underbrace{\frac{1}{2} \begin{bmatrix}
         1 & 1  \\
         1 & -1  \\
      \end{bmatrix}}_{T}
      \begin{bmatrix}
          V \\
          W \\
      \end{bmatrix}
   \]
   Here \(|\det{J_T}| = \frac{1}{2}\). 
   Consequently,  
   \[
      f_{V,W}(v,w) = \frac{1}{\pi \sqrt{v^2 - w^2} } e^{-\frac{1}{2}v}
   \]
   Becareful about the region, the region should be \( \{(v,w) \in R^+ \times R^+, v \geq w\}\) 
   \\ 
   You can check that 
   \[
      \int_0^\infty \int_0^v f_{V,W}(v,w)dwdv = 1  
   \]
    \item Are \(V\) and \(W\) independent? \\  
   Given the weird joint p.d.f., it's unlikely right?\\
   Because the joint p.d.f. is quite complicated for the marginal of \(W\), we can use conditional p.d.f. to check. 
   \\i.e. Given \(v\) such that \(f_V(v) > 0\), if the conditional p.d.f. \(f_{W}(w | V = v) = \frac{f_{V,W}(v,w)}{f_V(v)}\) is not a function of solely \(w\), then \(V,W\) are  not independent. 
   \\Since \(f_V(v) = \frac{1}{2}e^{-\frac{v}{2}} > 0\), we have 
   \[
      f_W(w| V = v) = \frac{2}{\pi \sqrt{v^{2}-w^2} }, 
   \] 
   which is a function of both \(v\) and \(w\). That means \(W\) and \(V\) are not independent.  
\end{itemize}


\end{document}