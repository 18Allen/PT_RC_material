\chapter{RC 2}
\recitation{1}{13 Oct. 18:30}{}

\section{Review: Random Variables}
(\cite{Und_Chatterjee} chapter 2) 
It is a 'function' mapping $\Omega$ to some number, i.e
\[
    X : \Omega \to \mathbb{R}
\]
\textbf{Notation}  
\begin{itemize}
    \item \(X \in A,\quad A \subseteq \mathbb{R}\): The set containing  \(\omega \in \Omega \) such that \(X(\omega ) \in B\). \(X \in B  = \{\omega  \in \Omega | X(\omega ) \in B \}\) 
\end{itemize}
\subsection{Discrete Random Variables}
The \textbf{range} of \(X\) is finite or countable. 
\begin{itemize}
    \item \textbf{probability mass function}: A measure on \(x_i\) given by the measure in \(\Omega \)
    \[
        f(x_i) = P(X = x_i)
    \] 
    \item Because \(X \in x_i\) for \(i = 1,2,\dots\) from a disjoint partition (Why?) of \(\Omega \), we have 
    \[
        \sum_{i=1}^{\infty} f(x_i) = P(\Omega ) = 1
    \]
    TBD(Panchenko assume that \(\Omega \) is countable )    
    \item Independence of random variables: 
    (We won't go through the detail in class, \textbf{strongly recommend} you to check out \cite{Und_Chatterjee} Proposition 3.) 
    It comes from the Independence of events(\ref{rc1_indep}). This is a theorem showing you the equivalent way of defining independence. 
    One using subcollection, one using collection of all. 
    
    Additional note: An infinite sequence of r.v \(X_1, X_2 , \dots\) is called Independence if \textbf{for any n}, \(X_1, X_2, \dots, X_n\) are independent.   
\end{itemize}
\section{Some Discrete r.v}
\begin{itemize}
    \item \textbf{Bernoulli r.v} \(\text{Bernoulli}(p) \)  
    \item \textbf{Binomial r.v} \(\text{binomial}(n,p) \)   
    TBD calculate the expectation of bin
    \item \textbf{Poisson r.v} \(\text{Poisson}(\lambda )\) 
       
    \item \textbf{Geometric r.v} \(\text{Geometric}(p) \)  
\end{itemize}
The most important thing to remember about a r.v (before a test) are 1. PMF; 2. expectation; 3. Varaiance. Examples about the r.v or the relation with other r.v's are helpful, too. You can those info in \cite{Gravner2021} Chapter 5.\\

\begin{exercise}[Poisson random variables]
    Suppose $X_n\sim\text{Pois}(\lambda)$ are i.i.d. Then\\
    1. Define $S_n=X_1+\cdots+X_n$. Show that $S_n\sim\text{Pois}(n\lambda)$.\\
    2. 
\end{exercise}

% From Bernoulli to Binomial 
\subsection{From Bernoulli to Binomial}
% From Binomial to Poisson
\begin{exercise}[Peak of Binomial distribution]
    (Feller p.59 q.5) The probability \(p_{k}\) that a given  cell contains exactly k balls (with a total \(r\) balls and \(n\) boxes) is given by the binomial distribution. 
    Show that the most probable number \(v\) satisfies 
    \[
        \frac{r+n-1}{n} \leq  v \leq \frac{r+1}{n}
    \]
    (In other words, it is asserted that \(p_0 < p_1 < \dots < p_{v-1} \leq p_{v} >  p_{v+1} \dots > p_r \) )
\end{exercise}
(Hint: Don't use derivative to find local extremum).

\subsection{From Binomial to Poisson}


\subsection{Estimate the error bound}


% The error bound Theorem 1.1
% Example of Poisson
% Panchenko Lemma 1.2 as practice of adv. cal. (Why do we need to have integrable)
% Or stability of Poisson