\documentclass[12pt]{article}

\usepackage[english]{babel}

\usepackage[letterpaper,top=2cm,bottom=2cm,left=3cm,right=3cm,marginparwidth=1.75cm]{geometry}

% Useful packages
\usepackage{amsmath}
\usepackage{amsfonts}
\usepackage{dsfont}
\usepackage{graphicx}
\usepackage[colorlinks=true, allcolors=blue]{hyperref}
%\usepackage[legalpaper, landscape, margin=1in]{geometry}
\usepackage{geometry}
\usepackage{cancel}
\geometry{
a4paper,
left=25mm,
right=25mm,
top=15mm,
}
\usepackage{setspace}
\setstretch{1.25}

\usepackage{pgfplots}
\pgfplotsset{every axis/.append style={
    axis x line=middle,    % put the x axis in the middle
    axis y line=middle,    % put the y axis in the middle
    axis line style={<->}, % arrows on the axis
    xlabel={$x$},          % default put x on x-axis
    ylabel={$y$},          % default put y on y-axis
    },
    cmhplot/.style={color=red,mark=none,line width=1pt,<->},
    soldot/.style={color=red,only marks,mark=*},
    holdot/.style={color=red,fill=white,only marks,mark=*},
}

\tikzset{>=stealth}

\begin{document}

\noindent
\section*{Quiz 4 Solution}
\begin{enumerate}
\item This is a problem about change of variable (CoV) for single variable 
\begin{itemize}
    \item[(a)] We may copy the c.d.f. method in HW8. First, observe that the range of \(U\) is \([0,\infty)\) not \(\mathbb{R}\) now.   
    
    To use the c.d.f. method, we consider 
    \[
        P(U \leq u) = P(X^2 \leq  u) = P(-\sqrt{u} \leq  X \leq \sqrt{u} ) = \int_{-\sqrt{u} }^{\sqrt{u}} \frac{1}{\sqrt{2\pi } }e^{-\frac{x^2}{2}}dx 
    \]
    The integral does not have analytical form, but that's fine. We only want its derivative \(f_{U}(u) \). By Leibniz integral rule 
    \[
        \frac{d}{du}  \int_{-\sqrt{u} }^{\sqrt{u}} \frac{1}{\sqrt{2\pi } }e^{-\frac{x^2}{2}}dx = \frac{1}{2\sqrt{u}}( \frac{1}{\sqrt{2\pi } }e^{-\frac{u}{2}})  -\frac{-1}{2\sqrt{u}}( \frac{1}{\sqrt{2\pi} }e^{-\frac{u}{2}})  =  \frac{1}{\sqrt{2\pi u } }e^{-\frac{u}{2}}
    \]
    \hspace{\textwidth}\(\square \) 
    \item[(b)] By definition of expectation 
    \[
        E[U] = \int_0^\infty u f_U(u)du = \frac{1}{\sqrt{2\pi } }\int_0^{\infty} u^{\frac{1}{2}} e^{-\frac{u}{2}} du \underbrace{=}_{u = 2t^2}\frac{4}{\sqrt{\pi } }\int_0^{\infty} t^2 e^{-t^2} dt   
    \]
    At this point you either recognize the integral as \(\Gamma (\frac{1}{2}) = \sqrt{\pi } \), or do integrate by part as follows 
    \[
        \int_0^{\infty} t^2e^{-t^2}dt = \cancelto{0}{-\frac{t}{2}e^{-t^2} |_{0}^{\infty}} + \frac{1}{2}\int_0^{\infty} e^{-t^2}dt  = \frac{\pi}{4} 
    \] 
    Hence, \(E[U] = 1\). 

    \textbf{Note:} If you notice \(E[U] = E[X^2]\), which is given to be \(1\) since \(1 = Var(X) = E[X^2] = E[X]^2 = E[X^2]\).     
    \hspace{\textwidth}\(\square \) 
\end{itemize}
\item This is a problem that applies c.d.f. method, and that's why the problem is designed this way. 
\begin{itemize}
    \item[(a)]
    First step, we check the joint p.d.f. of \(X\) and \(Y\). Since they are independent, the joint p.d.f. is the product of their respective p.d.f., i.e. 
    \[
        f(x,Y) = \lambda ^{2} e^{-\lambda (x+y)}
    \] 
    Then, we compute the c.d.f. of \(Z\).  
    \begin{equation*}
        \begin{aligned}
        P(Z \leq z ) &= P(X \leq zY) = \int_{0}^{\infty} \int_0^{zy}\lambda^{2} e^{-\lambda (x+y)}dxdy   
                \\ &= \int_0^{\infty} \lambda e^{- \lambda  y} (1-e^{-\lambda zy})dy 
                \\ &= 1  - \frac{1}{z+1}
        \end{aligned}
    \end{equation*}
    \hspace{\textwidth}\(\square \) 
    \item[(b)] 
    Getting p.d.f. from (a) is just a derivative away. Still, you have to be careful about the range of \(Z\). 
    In this case \(Z\) takes value in \([0,\infty )\) (or \(\mathbb{R}^+\) ).     
    \[
        f_Z(z) = P^\prime (Z \leq z) = \frac{1}{(z+1)^{2} }
    \]
    \hspace{\textwidth}\(\square \) 
\end{itemize}
\item 
\begin{itemize}
    \item[(a)] 
    This is a classic multi-variate change of variable problem. We will start with (if possible) finding \(T\) such that \((X,Y) = T(V,W)\). 
    \[
        \begin{bmatrix}
             X \\
             Y \\
        \end{bmatrix} = \underbrace{\frac{1}{2} \begin{bmatrix}
            1 &  1 \\
            1 &  -1 \\
        \end{bmatrix}}_{A}\begin{bmatrix}
             V \\
             W \\
        \end{bmatrix}
    \]
     In this case we have $T$ as a transformation matrix \(A\). (b.t.w. it means \(\left\vert \frac{\partial (x,y)}{\partial (v,w)} \right\vert  = \frac{1}{2}\) ) 
    
    Then, since \(X,Y\) are i.i.d. standard normal, the joint p.d.f. is   
    \[
        f_(x,y)  = \frac{1}{2\pi }e^{-\frac{x^{2}+y^2 }{2}}
    \]

    Finally, by multi-variate CoV (Please refer to HW8\_sol), the joint p.d.f. of \((V,W)\), \(g\) is  
       \[
        g(v,w) = f(T(v,w))\left\vert \frac{\partial (x,y)}{\partial (v,w)} \right\vert 
       \]
    Which is 
    \[
        g(v,w) = \frac{1}{4 \pi }e^{-\frac{(\frac{v+w}{2})^2 + (\frac{v-w}{2})^2}{2}}  
            = \frac{1}{4 \pi }e^{-\frac{v^2 +w^2}{4}} 
    \]
    \hspace{\textwidth}\(\square \) 
    \item[(b)]
    If both variables have p.d.f. and they also have joint p.d.f. (which is the case we have here), then \(V\) \(W\) are independent if \(g(v,w) = f_Z(z)f_{W}(w)\). 
    \[
        g(v,w)  = \frac{1}{4 \pi }e^{-\frac{v^2 +w^2}{4}}  = \underbrace{\frac{1}{2\pi \sqrt{2} }e^{-\frac{v^2}{2\cdot 2}}}_{f_Z(z)} \cdot \underbrace{\frac{1}{2\pi \sqrt{2} }e^{-\frac{w^2}{2\cdot 2}}}_{f_W(w)}
    \]  
    Hence, \(V,W\) are independent. 
    \hspace{\textwidth}\(\square \) 
    \item[(c)]
    The problem is equivalent to asking the marginal p.d.f. of \(Z\). 
    Due to the independence of \(V,W\) (as shown in (b)), this is just \(f_Z(z)\).     
    
    \textbf{Note:} Since \(X,Y\) are independent normal variables, we may also use the \textbf{stability of normal r.v.} to get \(X+Y \sim N(0,1+1)\) directly.   
\end{itemize}
\item 
This problem indents to find the transformation of \(X\), \(A\) , such that the new \textbf{covariance matrix} is \(\Sigma \).  
Since 
\[
    \Sigma  = AA^{\top}  
\]
is a semi-defninate matrix. It is good to start from eigenvalue decomposition. 
\[
    \Sigma = \begin{bmatrix}
        1 &  1 \\
        1 &  -1 \\
    \end{bmatrix}\begin{bmatrix}
        2 &  0 \\
        0 &  3 \\
    \end{bmatrix} \begin{bmatrix}
        1 &  1 \\
        1 &  -1 \\
    \end{bmatrix} = \begin{bmatrix}
        1 &  1 \\
        1 &  -1 \\
    \end{bmatrix}\begin{bmatrix}
        \sqrt{2}  &  0 \\
        0 &  \sqrt{3}  \\
    \end{bmatrix} 
    \begin{bmatrix}
        \sqrt{2}  &  0 \\
        0 &  \sqrt{3}  \\
    \end{bmatrix} 
    \begin{bmatrix}
        1 &  1 \\
        1 &  -1 \\
    \end{bmatrix}
\]
Hence, we may set \(A = \begin{bmatrix}
        1 &  1 \\
        1 &  -1 \\
    \end{bmatrix}\begin{bmatrix}
        \sqrt{2}  &  0 \\
        0 &  \sqrt{3}  \\
    \end{bmatrix} 
\) 
\\
\hspace{\linewidth}\(\square \) 
\item 
This is a rather weird question.  Observe that we have 
\begin{enumerate}
    \item \(f_X(x) = 1, x \in [0,1]\)
    \item \(P(X \leq  x) = x, x\in [0,1]\) 
    \item \(f_Y(y) = e^{-y} , y \in \mathbb{R}^+\) 
    \item \(P(Y \leq y) = 1 - e^{-y} , y \in \mathbb{R}^+\) 
\end{enumerate} 
Can we use c.d.f. method on this one? Let's try 
\[
    1 - e^{-y} = P(Y \leq y) = P(g(X) \leq y) = P(X \leq g^{-1}(y)) = g^{-1}(y)
\]
(About the existance of inverse function, you may first assume it and check that $g^{-1}$ increase with $y$ continuously, or you can first observe that \(P(X \leq g^{-1}(y)) \) )

Anyway, $x = g^{-1}(y) = 1 - e^{-y}$ gives us 
\[
    y = g(x) = -\ln{(1-x)} 
\] 
\hspace{\textwidth}\(\square \) 
\end{enumerate}
\end{document} 