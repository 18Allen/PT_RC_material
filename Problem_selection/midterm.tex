\documentclass[12pt]{article}

\usepackage[english]{babel}

\usepackage[letterpaper,top=2cm,bottom=2cm,left=3cm,right=3cm,marginparwidth=1.75cm]{geometry}

% Useful packages
\usepackage{amsmath}
\usepackage{amsfonts}
\usepackage{dsfont}
\usepackage{graphicx}
\usepackage[colorlinks=true, allcolors=blue]{hyperref}
%\usepackage[legalpaper, landscape, margin=1in]{geometry}
\usepackage{geometry}
\geometry{
a4paper,
left=25mm,
right=25mm,
top=15mm,
}
\usepackage{setspace}
\setstretch{1.25}

\begin{document}
\section*{Problems for midterm}
\subsection*{Chapter 1.}
\begin{itemize}
    \item \textbf{1. (Birthday problem)}\\
    Suppose there are \(n\) students in a class, and each has birthday equally likely to be 1 of 365 days (no leap year). 
    Write down the expression of probability that there exists at least a pair of student that share the same birthday\    
    \item \textbf{2. (Probability axiom)}\\
    (Modified from Panchenko Exercise 1.2.15)
Let \(\Omega  = \{a,b,c,d\}\) and the probability function \(P: \Omega \to [0,1]\). 
Suppose \(P(\{a,b\}) = 0.6, P(\{b,c\}) = 0.3, P(\{c,d\}) = 0.4\), Calculate \(P(\{a\}), P(\{b\}), P(\{c\}), P(\{d\})\)
    \item \textbf{3. (independence)}\\
We roll a die three times. Let \(A_{ij}\) be the event that the ith and jth rolls produce the same number. 
Show that the events \(A_{12}, A_{23}, A_{13}\) are pairwise independent but not independent events.
\end{itemize}

\subsection*{Chapter 2.}
\begin{itemize}
    \item \textbf{1. (Poisson $+$ Conditioning)} In your pocket there is a random number $N$ of coins, where $N$ has the Poisson distribution with parameter $\lambda$. You toss each coin once, with heads showing with probability $p$ each time.\\
    (a) Compute $\mathbb{P}(H=h\mid N=n)$, where $H$ is the total number of heads. \\
    (b) Show that the total number of heads has the Poisson distribution with parameter $\lambda p$.
    \item \textbf{2. (Refer to Homework 4.4.)} You and your opponent both roll a fair die. If one get a greater number than the other one, and that number $>3$, then the game ends and whoever rolls the larger number wins. Otherwise, we repeat the game.\\
    (a) Let $N$ be the number of rounds in this game. Write down the p.m.f. of $N$. \\
    \textbf{Solution (a).} $\mathbb{P}(N=n)=(1-p)^{n-1}p$ where $p=\frac{2}{3}$.
    $\hspace{\fill}\square$\\
    (b) What is $P(\text{you win})$? 
    \item \textbf{3. (Normalize constant)}\\
Consider a function \(f\) defined on \(\textbf{2},3,4,\dots\) such that \(f(x) = C \frac{1}{x(x+1)}\), where \(C\) is a constant. Please find \(C\) such that \(f\) is a pmf. 


\end{itemize}

\subsection*{Chapter 3.}
\begin{itemize}
    \item \textbf{1. (Birthday problem II.)}
    Suppose there are \(n\) students in a class, and each has birthday equally likely to be 1 of 365 days (no leap year). 
    What is the expectation of number of distinct birthday?
    \item \textbf{2. (Expectation and variance of matchings)} Let $S_n$ denotes the number of matchings of a random permutation of $n$ cards. Compute $\mathbb{E}(S_n)$ and $Var(S_n)$.
    \item \textbf{3. (Refer to Homework 6.11.)} Let $(X_i)_{1\leq i\leq n}$ be a sequence $n$ $i.i.d.$ random variables with
\begin{equation*}
\mathbb{P}(X_i=1)=\mathbb{P}(X_i=-1)=\frac{1}{2}.
\end{equation*}
Define $S_k=X_1+X_2+\cdots+X_k$ for $1\leq k\leq n$ as the $k$-th partial sum.\\
(a) Let $N$ be a random variable taking values from $\{1,\cdots,n\}$ with equal probability, independent to $(X_i)_{1\leq i\leq n}$. What is the variance of the random sum, $Var(S_N)$? \\
(b) Let $M$ be a random variable that has the same distribution as $N$ in (a), but independent to $N$ and $(X_i)_{1\leq i\leq n}$. What is $Cov(S_N,S_M)$? (You may encounter the calculation of $1^2+2^2+\cdots+(k-1)^2=\frac{k(k-1)(2k-1)}{6}$)\\
\textbf{Solution (a).} Write $S_N=S_N\mathds{1}_{\{N=1\}}+\cdots+S_N\mathds{1}_{\{N=n\}}$, then by linearity of expectation,
\begin{equation*}
\begin{aligned}
\mathbb{E}\big(S_N^2\big) &= \sum_{k=1}^n\mathbb{E}\big(S_N^2\mathds{1}_{\{N=k\}}\big)=\sum_{k=1}^n\mathbb{E}\big(S_k^2\mathds{1}_{\{N=k\}}\big)
\end{aligned}
\end{equation*}
Since $(X_i)_{1\leq i\leq n}$ and $N$ are independent, by RHS we have
\begin{equation*}
\mathbb{E}\big(S_N^2\big)=\sum_{k=1}^n\mathbb{E}\big(S_k^2\big)\mathbb{E}(\mathds{1}_{\{N=k\}})=\sum_{k=1}^nk\mathbb{P}(N=k)=\sum_{k=1}^n\frac{k}{n}=\frac{n+1}{2}.
\end{equation*}
It's easy to see $\mathbb{E}(S_N)=0$ by  this method. Hence $Var(S_N)=\mathbb{E}\big(S_N^2\big)=\frac{n+1}{2}$.
$\hspace{\fill}\square$\\
\textbf{Solution (b).} Similarly, write $S_NS_M=S_NS_M\mathds{1}_{\{N=M\}}+S_NS_M\mathds{1}_{\{N\neq M\}}$ and compute the expectation respectively. Since $N$ and $M$ are $i.i.d.$,
\begin{equation*}
\mathbb{E}(S_NS_M\mathds{1}_{\{N\neq M\}})=\mathbb{E}(S_NS_M\mathds{1}_{\{N < M\}})+\mathbb{E}(S_NS_M\mathds{1}_{\{N > M\}})=2\mathbb{E}(S_NS_M\mathds{1}_{\{N<M\}}),
\end{equation*}
and it easy to see that $\mathds{1}_{\{N<M\}}=\sum_{i=1}^{n-1}\sum_{j=i+1}^n\mathds{1}_{\{N=i,\,M=j\}}$, thus
\begin{equation*}
\begin{aligned}
\mathbb{E}(S_NS_M\mathds{1}_{\{N<M\}}) &= \sum_{i=1}^{n-1}\sum_{j=i+1}^n\mathbb{E}(S_NS_M\mathds{1}_{\{N=i,\,M=j\}}) \\&
=\sum_{i=1}^{n-1}\sum_{j=i+1}^n\mathbb{E}(S_iS_j\mathds{1}_{\{N=i,\,M=j\}}) \\&
=\sum_{i=1}^{n-1}\sum_{j=i+1}^n\mathbb{E}(S_iS_j)\mathbb{E}(\mathds{1}_{\{N=i,\,M=j\}}) \\&
=\sum_{i=1}^{n-1}\sum_{j=i+1}^n\mathbb{E}(S_iS_j)\mathbb{P}(N=i,M=j).
\end{aligned}
\end{equation*}
Notice that since $(X_i)_{1\leq i\leq n}$ is an $i.i.d.$ sequence,
\begin{equation*}
\mathbb{E}(S_iS_j)=\mathbb{E}(S_i^2)+\mathbb{E}(S_i(X_{i+1}+\cdots +X_j))=\mathbb{E}(S_i^2)+\mathbb{E}(S_i)\mathbb{E}(X_{i+1}+\cdots +X_j),
\end{equation*}
which is $\mathbb{E}(S_i^2)+0=i$. So the summation above becomes
\begin{equation*}
\mathbb{E}(S_NS_M\mathds{1}_{\{N<M\}})=\sum_{i=1}^{n-1}\frac{(n-i)i}{n^2}=\frac{n-1}{2}-\frac{n(n-1)(2n-1)}{6n^2}=\frac{n^2-1}{6n}.
\end{equation*}
So $\mathbb{E}(S_NS_M\mathds{1}_{\{N\neq M\}})=\frac{n^2-1}{3n}$. Together with
\begin{equation*}
\mathbb{E}(S_NS_M\mathds{1}_{\{N=M\}})=\sum_{i=1}^n\mathbb{E}(S_i^2)\mathbb{P}(N=i,M=i)=\frac{n(n+1)}{2n^2},
\end{equation*}
we have
\begin{equation*}
\mathbb{E}(S_NS_M)=\frac{n}{3} + \frac{1}{2}+\frac{1}{6n}=Cov(S_N,S_M).
\end{equation*}
$\hspace{\fill}\square$
\end{itemize}


%\newpage
%\section*{Homework 6.}
%$\textbf{6.11.}$ Use the method of indicators, for $i\neq j$, we can write 
%\begin{equation*}
%\begin{aligned}
%\mathbb{E}(X_{e(i)}X_{e(j)}) &= \mathbb{E}(X_{e(i)}X_{e(j)}\mathds{1}_{\{e(i)\neq e(j)\}}+X_{e(i)}X_{e(j)}\mathds{1}_{\{e(i)= e(j)\}}) \\&
%=\mathbb{E}(X_{e(i)}X_{e(j)}\mathds{1}_{\{e(i)\neq e(j)\}})+\mathbb{E}(X_{e(i)}X_{e(j)}\mathds{1}_{\{e(i)= e(j)\}}).
%\end{aligned}
%\end{equation*}
%Adapt the indicator method again, you can calculate
%\begin{equation*}
%\begin{aligned}
%\mathbb{E}(X_{e(i)}X_{e(j)}\mathds{1}_{\{e(i)\neq e(j)\}}) &= \sum_{k\neq i;\;l\neq j;\; k\neq l}\mathbb{E}(X_{e(i)}X_{e(j)}\mathds{1}_{\{e(i)=k,\,e(j)=l\}}) \\&
%=\sum_{k\neq i;\;l\neq j;\; k\neq l}\mathbb{E}(X_k X_l\mathds{1}_{\{e(i)=k,\,e(j)=l\}}) \\&
%=\sum_{k\neq i;\;l\neq j;\; k\neq l}\mathbb{E}(X_k)\mathbb{E}(X_l)\mathbb{E}(\mathds{1}_{\{e(i)=k\}})\mathbb{E}(\mathds{1}_{\{e(j)=l\}}) \\&
%=0,
%\end{aligned}
%\end{equation*}
%and
%\begin{equation*}
%\begin{aligned}
%\mathbb{E}(X_{e(i)}X_{e(j)}\mathds{1}_{\{e(i)=e(j)\}}) &= \sum_{k\neq i;\;k\neq j}\mathbb{E}(X_{e(i)}X_{e(j)}\mathds{1}_{\{e(i)=k,\,e(j)=k\}}) \\&
%=\sum_{k\neq i;\;k\neq j}\mathbb{E}(X_k^2\mathds{1}_{\{e(i)=k,\,e(j)=k\}}) \\&
%=\sum_{k\neq i;\;k\neq j}\mathbb{E}(X_k^2)\mathbb{E}(\mathds{1}_{\{e(i)=k\}})\mathbb{E}(\mathds{1}_{\{e(j)=k\}}) \\&
%=\sum_{k\neq i;\;k\neq j}1\cdot\mathbb{P}(e(i)=k)\mathbb{P}(e(j)=k) \\&
%=\sum_{k\neq i;\;k\neq j}\frac{1}{(n-1)^2} = \frac{n-2}{(n-1)^2}.
%\end{aligned}
%\end{equation*}
%Hence $\mathbb{E}(X_{e(i)}X_{e(j)})=\frac{n-2}{(n-1)^2}$ for $i\neq j$.
%
%When $i=j$, $\mathbb{E}\big(X_{e(i)}^2\big)=\sum_{k\neq i}\mathbb{E}(X_{e(i)}^2\mathds{1}_{\{e(i)=k\}})=\sum_{k\neq i}\mathbb{E}(X_k^2\mathds{1}_{\{e(i)=k\}})$. Use the independence calculation again, you can see that $\mathbb{E}\big(X_{e(i)}^2\big)=(n-1)\cdot1\cdot\frac{1}{(n-1)}=1$.
%
%Then the variance can be computed as
%\begin{equation*}
%\begin{aligned}
%Var(X_{e(1)}+\cdots+X_{e(n)}) &= \sum_{i,j}Cov(X_{e(i)}, X_{e(j)})=\sum_{i,j}\mathbb{E}(X_{e(i)}X_{e(j)})-\mathbb{E}(X_{e(i)})\mathbb{E}(X_{e(j)}) \\&
%=\sum_{i,j}\mathbb{E}(X_{e(i)}X_{e(j)})
%=\sum_{i=j}1+\sum_{i\neq j}\frac{n-2}{(n-1)^2}=n+\frac{n(n-2)}{n-1}.
%\end{aligned}
%\end{equation*}
%You can check that $\mathbb{E}(X_{e(i)})=0$ with the similar method.
%$\hspace{\fill}\square$


\end{document}