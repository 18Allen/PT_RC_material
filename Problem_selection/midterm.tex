\documentclass[12pt]{article}

\usepackage[english]{babel}

\usepackage[letterpaper,top=2cm,bottom=2cm,left=3cm,right=3cm,marginparwidth=1.75cm]{geometry}

% Useful packages
\usepackage{amsmath}
\usepackage{amsfonts}
\usepackage{dsfont}
\usepackage{graphicx}
\usepackage[colorlinks=true, allcolors=blue]{hyperref}
%\usepackage[legalpaper, landscape, margin=1in]{geometry}
\usepackage{geometry}
\geometry{
a4paper,
left=25mm,
right=25mm,
top=15mm,
}
\usepackage{setspace}
\setstretch{1.25}

\begin{document}
\section*{Problems for midterm}
\subsection*{Chapter 1.}
\textbf{1. (Birthday Paradox)}\\
Suppose   Write down the expression of probability that there exists at least a pair of student that share the same birthday
\textbf{2. (Probability axiom)}\\
(Modified from Panchenko Exercise 1.2.15)
Let \(\Omega  = \{a,b,c,d\}\) and the probability function \(P: \Omega \to [0,1]\). 
Suppose \(P(\{a,b\}) = 0.6, P(\{b,c\}) = 0.3, P(\{c,d\}) = 0.4\), Calculate \(P(\{a\}), P(\{b\}), P(\{c\}), P(\{d\})\)
\\
\textbf{3. ()} 
 
\subsection*{Chapter 2.}
\textbf{1. (Poisson $+$ Conditioning)} In your pocket there is a random number $N$ of coins, where $N$ has the Poisson distribution with parameter $\lambda$. You toss each coin once, with heads showing with probability $p$ each time. Show that the total number of heads has the Poisson distribution with parameter $\lambda p$. \\ \\

\textbf{2. (Modified by Homework 4.4.)} You and your opponent both roll a fair die. If one get a greater number than the other one, and that number $>3$, then the game ends and whoever rolls the larger number wins. Otherwise, we repeat the game. What is $P(\text{you win})$?
\textbf{3. (Normalize constant)}
Consider a function \(f\) defined on \(\textbf{2},3,4,\dots\) such that \(f(x) = C \frac{1}{x(x+1)}\), where \(C\) is a constant. Please find \(C\) such that \(f\) is a pmf. 

\subsection*{Chapter 3.}
\textbf{1. (Birthday)}
\\ \\
\textbf{2. (Expectation and variance of matchings)} Let $S_n$ denotes the number of matchings of a random permutation of $n$ cards. Compute $\mathbb{E}(S_n)$ and $Var(S_n)$.
\\ \\
\textbf{3. (Random sum)} Let $(X_i)_{1\leq i\leq n}$ be a sequence $n$ $i.i.d.$ random variables with
\begin{equation*}
\mathbb{P}(X_i=1)=\mathbb{P}(X_i=-1)=\frac{1}{2}.
\end{equation*}
Let $N$ be a random variable taking value from $\{1,\cdots,n\}$ with equal probability, independent to $(X_i)_{1\leq i\leq n}$. Define $S_k=X_1+X_2+\cdots+X_k$ for $1\leq k\leq n$.\\
(a) What is the variance of the random sum, $Var(S_N)$? \\
(b) Let $M$ be a random variable that has the same distribution as $N$, but independent to $N$ and $(X_i)_{1\leq i\leq n}$. What is $Cov(S_N,S_M)$? (You may encounter the calculation of $1^2+2^2+\cdots+(k-1)^2=\frac{k(k-1)(2k-1)}{6}$)



\newpage
\section*{Homework 6.}
$\textbf{6.11.}$ Use the method of indicators, for $i\neq j$, we can write 
\begin{equation*}
\begin{aligned}
\mathbb{E}(X_{e(i)}X_{e(j)}) &= \mathbb{E}(X_{e(i)}X_{e(j)}\mathds{1}_{\{e(i)\neq e(j)\}}+X_{e(i)}X_{e(j)}\mathds{1}_{\{e(i)= e(j)\}}) \\&
=\mathbb{E}(X_{e(i)}X_{e(j)}\mathds{1}_{\{e(i)\neq e(j)\}})+\mathbb{E}(X_{e(i)}X_{e(j)}\mathds{1}_{\{e(i)= e(j)\}}).
\end{aligned}
\end{equation*}
Adapt the indicator method again, you can calculate
\begin{equation*}
\begin{aligned}
\mathbb{E}(X_{e(i)}X_{e(j)}\mathds{1}_{\{e(i)\neq e(j)\}}) &= \sum_{k\neq i;\;l\neq j;\; k\neq l}\mathbb{E}(X_{e(i)}X_{e(j)}\mathds{1}_{\{e(i)=k,\,e(j)=l\}}) \\&
=\sum_{k\neq i;\;l\neq j;\; k\neq l}\mathbb{E}(X_k X_l\mathds{1}_{\{e(i)=k,\,e(j)=l\}}) \\&
=\sum_{k\neq i;\;l\neq j;\; k\neq l}\mathbb{E}(X_k)\mathbb{E}(X_l)\mathbb{E}(\mathds{1}_{\{e(i)=k\}})\mathbb{E}(\mathds{1}_{\{e(j)=l\}}) \\&
=0,
\end{aligned}
\end{equation*}
and
\begin{equation*}
\begin{aligned}
\mathbb{E}(X_{e(i)}X_{e(j)}\mathds{1}_{\{e(i)=e(j)\}}) &= \sum_{k\neq i;\;k\neq j}\mathbb{E}(X_{e(i)}X_{e(j)}\mathds{1}_{\{e(i)=k,\,e(j)=k\}}) \\&
=\sum_{k\neq i;\;k\neq j}\mathbb{E}(X_k^2\mathds{1}_{\{e(i)=k,\,e(j)=k\}}) \\&
=\sum_{k\neq i;\;k\neq j}\mathbb{E}(X_k^2)\mathbb{E}(\mathds{1}_{\{e(i)=k\}})\mathbb{E}(\mathds{1}_{\{e(j)=k\}}) \\&
=\sum_{k\neq i;\;k\neq j}1\cdot\mathbb{P}(e(i)=k)\mathbb{P}(e(j)=k) \\&
=\sum_{k\neq i;\;k\neq j}\frac{1}{(n-1)^2} = \frac{n-2}{(n-1)^2}.
\end{aligned}
\end{equation*}
Hence $\mathbb{E}(X_{e(i)}X_{e(j)})=\frac{n-2}{(n-1)^2}$ for $i\neq j$.

When $i=j$, $\mathbb{E}\big(X_{e(i)}^2\big)=\sum_{k\neq i}\mathbb{E}(X_{e(i)}^2\mathds{1}_{\{e(i)=k\}})=\sum_{k\neq i}\mathbb{E}(X_k^2\mathds{1}_{\{e(i)=k\}})$. Use the independence calculation again, you can see that $\mathbb{E}\big(X_{e(i)}^2\big)=(n-1)\cdot1\cdot\frac{1}{(n-1)}=1$.

Then the variance can be computed as
\begin{equation*}
\begin{aligned}
Var(X_{e(1)}+\cdots+X_{e(n)}) &= \sum_{i,j}Cov(X_{e(i)}, X_{e(j)})=\sum_{i,j}\mathbb{E}(X_{e(i)}X_{e(j)})-\mathbb{E}(X_{e(i)})\mathbb{E}(X_{e(j)}) \\&
=\sum_{i,j}\mathbb{E}(X_{e(i)}X_{e(j)})
=\sum_{i=j}1+\sum_{i\neq j}\frac{n-2}{(n-1)^2}=n+\frac{n(n-2)}{n-1}.
\end{aligned}
\end{equation*}
You can check that $\mathbb{E}(X_{e(i)})=0$ with the similar method.
$\hspace{\fill}\square$


\end{document}