\chapter{RC 4}
%\lecture{3}{22 Sep. 18:30}{Temp}
\recitation{4}{27 Oct. 18:30}{}

\section{Review}
\subsection{Variation}
Variance of a random variable \(X\) is the second moment of the demeaned \(X-E[X]\), i.e
\[
    \text{Var} (X) = E[(X-E[X])^2] = E[X^2] - E[X]^2
\]
\subsection{Covariance}
\begin{exercise}
    Find a counterexample of two discrete random variable \(X,Y\) such that \(\text{Cov}(X,Y) = 0  \), but they are not independent. 
\end{exercise}
Answer: \footnote[1]{\(Y\) with \(f_y(1) = f_y(-1) = f(0) = \frac{1}{3}\) (uniform), and \(X = |Y|\) } 
\section{The method of indication}
(\cite{Und_Chatterjee} p.33)" The method of indicators is a technique for evaluating the expected value/variance of a random variable by finding a way to write it as a sum of indicator function."

The \textbf{indicator} of \(A\) is a random variable, denoted by \(1_A\), (you can understand it as taking \(A\) or not), defined as follows  
\[
    1_A(\omega) = \begin{cases} 1, \quad \text{if } \omega \in A \\ 0,\quad \text{if } \omega \notin A \end{cases}
\]

If \(X\) can be represented as \(X = \sum_{i=1}^{n} 1_{A_i}\), we can use \textbf{linearity of expectation} to write 
\[
    E[X] = \sum_{i=1}^{n} E[1_{A_i}]
\]  
(Remember here we don't have to worry about independence of \(1_{A_i}\) .)
\subsection{For expectation}
\begin{eg}[Coin Run ].\\
   \begin{itemize}
       \item A biased coin is tossed \(n\)  times, and heads shows with probability p on each toss. A run is a
sequence of throws which result in the same outcome, so that, for example, the sequence 
\[
HHTHTTH
\]    
contains five runs. Show that the expected number of runs is \(1 + 2(n - l)p(1 - p) \). Find the variance
of the number of runs. 
        \item As a review, what is the expected length of a run. 
        \item A head run is simply a continuous sequence of heads. Consider the first problem but with head run.
   \end{itemize} 
\end{eg}
\begin{exercise}
    Of the \(2n\)  people in a given collection of \(n\)  couples, exactly \(m\)  die. Assuming that the m have
been picked at random, find the mean number of surviving couples. This problem was formulated by
Daniel Bernoulli in 1768.\\
\end{exercise}
Hint: Find indicator on the survival for each couple.
%\subsection{For variance}
