\chapter{RC 0}
%\lecture{3}{22 Sep. 18:30}{Temp}
\recitation{0}{22 Sep. 18:30}{}
\section{Basic rules}
\begin{itemize}
    \item Recitation: Thursday 12:30 to 13:30, 18:30 to 19:30
    \item My office hour is right after recitation.
    \item Grade distribution: (quiz correction + attend rc): $1\times 4$; mid correction + attend rc:  $2$; Attend one rc: $4$ 
\end{itemize}
\section{Review of some high school comminatory and probability tricks}
Early chapters are about high school counting things over again.
We will swiftly go through the concept.  
\subsection{Permutation and Combination}
\begin{itemize}
    \item Permutation\\
		\#ways to form an ordering of \(m\) out of \(n\) different things. 
	\[
		P(n,m)
	\]
    \item Combination\\
		\#ways to form a group of \(m\) with \(n\) different things.
	\[
		C(n,m),\quad \begin{pmatrix}
			 n \\
			 m \\
		\end{pmatrix}
	\]
    \item Multinomial coefficents \cite*{Gravner2021}\\
		\#ways to divide a set of n elements into \(r\) (distinguishable) subsets of \(n_1, n_2 , \dots, n_r \) elements.
		\[
			\frac{n!}{n_1! n_2! \dots n_r!}
		\]
\end{itemize}
\subsection{Set Operation}
\begin{itemize}
    \item De Morgan's Law
\end{itemize}
\section{Axioms of Probability}
\subsection{Probability Space}
The \textbf{probability space} is a triple \(\Omega,\mathcal{\MakeUppercase{f}},P\) that contains  
\begin{itemize}
	\item The \textbf{sample space} \(\Omega \) contains all possible outcome.  
	\item The \(\sigma\)-algebra \(\mathcal{\MakeUppercase{f}} \) is the \textbf{event space}. It is a subset of the power set of \(\Omega \) we are interested in. 
	\item The \textbf{probability measure} \(P\) is a function \(P:\mathcal{\MakeUppercase{f}} \to [0,1] \) that satisfies the three axioms. 
	\begin{enumerate}
		\item \(P(\Omega ) = 1\) 
		\item Non-negative
		\item Countable additivity for disjoint sets in \(\mathcal{\MakeUppercase{f}} \).  
	\end{enumerate}
\end{itemize}
\subsection{Standard process}
A standard process of solving these problems (HW1,2) is to find the size of possible outcome first. 
Then, finding the size of desired event, and the ratio of the two is the prob. 


\begin{eg}
	\cite{Gravner2021} Example 3.11\\
You have 10 pairs of socks in the closet. Pick 8 socks at random. For every i,
compute the probability that you get i complete pairs of socks.\\
\begin{itemize}
	\item \# outcome:
	\item \# desirable outcome:
	\item the probability is : 
\end{itemize}
\end{eg}
\begin{eg}
	\cite{Gravner2021} Problem 3.2 (HW2 problem 2)\\
	Three married couples take seats around a table at random. Compute \(P(\text{no wife sits next to her husband} )\). 
	Use Inclusion-Exclusion principle to compute the probability of its complement event. 
\end{eg}

\subsection{Why do we need to set \(\sigma\)-algebra: Vitali set}
Have you ever wonder: Why would I need \(\mathcal{\MakeUppercase{f}} \) if I have \(\Omega\) already? As the textbook said, you won't have any problem with this notion. 
However, things get messy when we encounter set of infinity size. The idea of "length" will not be clear then. 
We use \textbf{Vitali set} \(V\) as an example on \(\mathbb{R}\) to show that we can't have a measure on  \(V\).
This problem is one of the reason that we only put probability measure on \(\mathcal{\MakeUppercase{f}} \).  
	

For more information:
\href{https://youtu.be/hcRZadc5KpI}{How the Axiom of Choice Gives Sizeless Sets | Infinite Series}

\section{Homework Help}
TBD