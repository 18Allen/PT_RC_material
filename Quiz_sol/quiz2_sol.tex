\documentclass[12pt]{article}

\usepackage[english]{babel}

\usepackage[letterpaper,top=2cm,bottom=2cm,left=3cm,right=3cm,marginparwidth=1.75cm]{geometry}

% Useful packages
\usepackage{amsmath}
\usepackage{amsfonts}
\usepackage{dsfont}
\usepackage{graphicx}
\usepackage[colorlinks=true, allcolors=blue]{hyperref}
%\usepackage[legalpaper, landscape, margin=1in]{geometry}
\usepackage{geometry}
\geometry{
a4paper,
left=25mm,
right=25mm,
top=15mm,
}
\usepackage{setspace}
\setstretch{1.25}

\usepackage{pgfplots}
\pgfplotsset{every axis/.append style={
    axis x line=middle,    % put the x axis in the middle
    axis y line=middle,    % put the y axis in the middle
    axis line style={<->}, % arrows on the axis
    xlabel={$x$},          % default put x on x-axis
    ylabel={$y$},          % default put y on y-axis
    },
    cmhplot/.style={color=red,mark=none,line width=1pt,<->},
    soldot/.style={color=red,only marks,mark=*},
    holdot/.style={color=red,fill=white,only marks,mark=*},
}

\tikzset{>=stealth}
\begin{document}

\noindent
{\bf{Quiz 2: Give yourself 50 minutes to solve 5 of the following 6 problems. Each problem weights 20 point scores}}
\begin{enumerate}

\item
Let X be a discrete random variable with the following p.m.f
\[
    P_X(x) = \begin{cases}
        0.3, \text{ for } x = 3 \\
        0.2, \text{ for } x = 5 \\
        0.3, \text{ for } x = 8 \\
        0.2, \text{ for } x = 10 \\
        0, \text{ otherwise}
    \end{cases}
\]
\begin{enumerate}
    \item Find the function \(F_X(x) = P(X \leq  x)\), for \(x \in \mathbb{N}\).
    
    \textbf{Solution.} 
    \[
        F_X(x) = \begin{cases} 0,  \text{ for } x = 1,2 \\
            0.3, \text{ for } 3 \leq x < 5 \\
            0.5, \text{ for } 5 \leq  x < 8 \\
            0.8, \text{ for } 8 \leq x < 10 \\
            1, \text{ for } x \geq  10
        \end{cases}, x \in \mathbb{N}
    \] 
    Other equivalent form is permitted, too.
    \item Plot the function \(F_X(x)\).\\
        \begin{tikzpicture}
            \begin{axis}[
                    xmin=0,xmax=11,
                    ymin=0,ymax=1.5,
                ]
                \addplot[cmhplot,-,domain=0:3]{0};
                \addplot[cmhplot,-,domain=3:5]{0.3};
                \addplot[cmhplot,-,domain=5:8]{0.5};
                \addplot[cmhplot,-,domain=8:10]{0.8};
                \addplot[cmhplot,->,domain=10:11]{1};
                \addplot[holdot]coordinates{(3,0)(5,0.3)(8,0.5)(10,0.8)};
                \addplot[soldot]coordinates{(3,0.3)(5,0.5)(8,0.8)(10,1)};
                \addplot[mark=none,black,dotted] coordinates {(0,1) (10,1)};
            \end{axis}
        \end{tikzpicture}
\end{enumerate}






\item
Let \(S = \{1,2,\dots, n\}\)  and suppose that \(A\) and \(B\) are, independently,
equally likely to be any of the \(2^n\) subsets (including the null set and \(S\) itself) of \(S\).

\begin{itemize}
    \item[(a)] Show that $ P(A \subset B) = (\frac{3}{4})^n$
    Hint: Let N(B) denote the number of elements in \(B\).
    Use $P (A \subset B )= \sum_{i=0}^{n} P(A \subset B \mid  |B| = i)P( |B| = i)$

    \textbf{Solution.}\\
    As suggested in the hint, it is convenient to break \(P(A \subset B, |B| = i)\) to the form of conditional probability \(P(A \subset B \mid  |B| = i)P( |B| = i) \). 
    
    The probability of \(B \subset S\) having size \(i\),   \(P(|B| = i )\),  is \(\begin{pmatrix}
         n \\
          i\\
    \end{pmatrix} / 2^n\). \\
    
    The probability of \(A \subset S\) being a subset of \(B\) given \(|B| = i\), \(P(A \subset B | |B| = i)\), is \(\frac{2^i}{2^n} = 2^{i-n}\)
    
    Hence, the answer is 
    \[
        \sum_{i=0}^{n} P(A \subset B | |B| = i) P(|B| = i) = 4^{-n} \underbrace{\sum_{i=0}^{n}\begin{pmatrix}
             n \\
             i \\
        \end{pmatrix}2^i}_{(1+2)^n} = (\frac{3}{4})^n 
    \]
    
    \textbf{Another Solution.}
    You can also do it without using conditional probability. Observe that the elements of \(S\) either in \(A (\text{, which is also } A \cap B), B-A \), or neither.   
    Therefore, we can think of the number of desired event as placing \(n\) distinct balls to 3 distinct buckets, which is \(3^n\). 
    Divide the number of possible \(A, B\) combination and we get the answer \((\frac{3}{4})^n\).    
    \item[(b)] Show that $P(A \cap B  = \emptyset) = (\frac{3}{4})^n$
    
    \textbf{Solution.} 
    \begin{equation*}
        \begin{aligned}
            P(A\cap B = \emptyset)  &= \sum_{i=0}^{n} P(A \subset B^c| |B| = i ) P(|B| = i ) \\
             &= \frac{1}{4^n} \underbrace{\sum_{i=0}^{n} 2^{n-i}\begin{pmatrix}
                 n \\
                 i \\
             \end{pmatrix}}_{(2+1)^n} \\
             &= (\frac{3}{4})^n
        \end{aligned}
    \end{equation*}
\hspace{\linewidth}\(\square  \) 
\end{itemize}

\item
When you bet on head in coin tossing, your chance of winning is \(\frac{1}{2}\).
Suppose also that bets over \$ 250 are not allowed.
You decide to play the following strategy: you start with a \$1 bet and double the bet until either you
win (the same amount as the bet) or the bet exceeds \$250;
then you start again with a \$1 bet and repeat. We will call
this sequence of bets in the strategy until restart with a \$1
bet ‘one round’. If you play 1000 rounds of this strategy

\begin{enumerate}
    \item How many rounds you have to win at least in 1000 rounds so that net revenue \(\geq 0\).
    
    \textbf{Solution.} \\ 
    Suppose in the $i$-th round, $1\leq i\leq 1000$, the money you'll get if end up winning at the $n$-th bet is
\begin{equation*}
-1-2-\cdots-2^{n-2}+2^{n-1}=2^{n-1}-(2^{n-1}-1)=1,
\end{equation*}
which is independent to $n$. Otherwise, if you end up lose all the bets, then the money you'll earn is
\begin{equation*}
-1-2\cdots-2^7=-(2^8-1) = -255.
\end{equation*}
Note that the maximum number of bets is $8$ times since the amount on the 9th bet is $2^{9-1} = 256>250$. 

Therefore, suppose you want to earn some money in 1000 rounds, the number of rounds you win/lose, name \(W\) and \(L\), satisfies 
\[
    \begin{cases}
        L + W  = 1000\\
        -255L + 1W \geq  0  .
    \end{cases}
\]  
Solving the system of equation you will get \(W \geq  996.09\dots \implies W \geq 997\) 
\hspace{\linewidth}\({\square}\) 


    \item Write down the expression of the probability that net revenue \(\geq 0\).

    \textbf{Solution.} 
    Clearly, you'll lose \$255 with probability $q=(1-p)^8$. Therefore, 
\[\mathbb{P}(\text{ends up getting $1$ dollar in a round})=1-q.\]

    The number of loses is the sum of 1000 Bernoulli random variable with probability \(q\). 
    Given the answer in (a), we have win at least 997 times, i.e lose less than 4 times. 
    Hence, the probability is 
    \[
        \sum_{i=0}^{3} \begin{pmatrix}
             1000 \\
             i \\
        \end{pmatrix}q^i(1-q)^{1000-i}
    \] 
        \hspace{\linewidth}\(\square\) 
\end{enumerate}




\item
\begin{enumerate}
\item For what value of $C$ do the function $C2^{x}/x!$  define a probability mass functions on $1,2,3,...$?

\textbf{Solution.}\\
If a function \(f\) with countable range $S$ meets the following conditions 
\begin{itemize}
    \item non-negative, \(f(x) \geq  0 , x \in S\)
    \item sum to 1, \(\sum_{x \in S} f(x) = 1 \)  
\end{itemize}
we call it a p.m.f \\
Here, the first condition is trivial, and in order for \(C2^x / x!\) to be a p.m.f, 
\[
    C \underbrace{\sum_{x \in \mathbb{N}} 2^x / x!}_{e^2 -1} = 1 
\] 
Hence, \(C = \frac{1}{e^2 -1}\) 

\hspace{\linewidth}\(\square\)   

\item
If X is geometric show that \(P(X = n + k | X > n) = P(X = k)\), for \(k,n \geq 1\).
\textbf{Solution} \\
We call Geometric random variable \textbf{memoryless} because of this. 
If \(X\) follows Geo(\(p\)), 
\begin{equation*}
   \begin{aligned}
    P(X = n+k | X > n) &= \frac{P(X = n+k, X > n)}{P(X > n)} = \frac{p(1-p)^{n+k-1}}{p(1-p)^{n} \frac{1}{p}} 
    \\ & = p(1-p)^{k-1} 
    \\ &= P(X = k)
   \end{aligned} 
\end{equation*}
\hspace{\linewidth} $\square$  
\end{enumerate}

\item
\begin{enumerate}
\item
Let $X$ and $Y$ be discrete random variables with the joint probability mass function $f(x,y)$. Assume that $f(x,y)=g(x)k(y)$ for some
probability mass functions $g(x)$ and $k(y)$. Prove that $X$ and $Y$ are independent.
\item In a sequence of tosses of a p-coin, let $X$ be the number of tosses required to get the first head and $Y$ be the number of tosses to get the second head after getting the first head. Show that $X$ and $Y$ are independent Geo(p) random variables.
\end{enumerate}

\item
(De Moivre trials) Each trial may result in any of \(t\) given outcomes (\(t\) is a fixed positive integer). The ith outcomes having probability \(p_i\), and \(\sum_{i=1}^{t} p_i  = 1\).
Let the number of occurrences of the ith occurances of the ith outcome in n independent trials. The probability is as the following
\[
    P(N_i = n_i, \text{ for all } 1 \leq i \leq t) = \frac{n!}{n_1!n_2!\cdots n_t!}p_1^{n_1}p_2^{n_2} \cdots p_t^{n_t}
\]
for any collection \(n_1, n_2, \dots, n_t\) of non-negative integers with sum \(n\). The vector \(N\) is said to have the multinomial distribution.
Calculate the marginal probability of \(N_1\), i.e calculate \(P(N_1 = n_1), n_1 = 0,1,\dots,n\).



\end{enumerate}

\end{document} 