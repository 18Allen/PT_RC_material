\documentclass[12pt]{article}

\usepackage[english]{babel}

\usepackage[letterpaper,top=2cm,bottom=2cm,left=3cm,right=3cm,marginparwidth=1.75cm]{geometry}

% Useful packages
\usepackage{amsmath}
\usepackage{amsfonts}
\usepackage{dsfont}
\usepackage{graphicx}
\usepackage[colorlinks=true, allcolors=blue]{hyperref}
%\usepackage[legalpaper, landscape, margin=1in]{geometry}
\usepackage{geometry}
\geometry{
a4paper,
left=25mm,
right=25mm,
top=15mm,
}
\usepackage{setspace}
\setstretch{1.25}

\begin{document}
\section*{Partial solution to HW 4 and HW 5}
\section*{HW4}
\subsection*{Q6}
The idea of disjoint union will be helpful to us.
Given an set \(A\in \mathbb{R}\), we may assume that \(P(X \in A) \geq  P(Y \in A)\). (Otherwise, replace X with Y in the following proof).  
The RHS can be written as
\begin{equation*}
\begin{aligned}
    \text{RHS} &=  1 - P(X = Y ) \\
             &= [ P(X \in A) + P(X \in A^c) ] - [P(X = Y, Y \in A) + P(X = Y, Y \in A^c)]\\
             &= [ P(X \in A) + P(X \in A^c) ] - [P(X = Y, Y \in A) + P(X = Y, X \in A^c)]\\
             & \geq P(X\in A) - P(Y \in A) + P(X\in A^c) - P(X\in A^c, X = Y) \\
             & = LHS + P(X\in A^c, X \neq Y) \\ 
             & \geq LHS 
\end{aligned}
\end{equation*}
\hspace{\textwidth}$\square$   

Another way of doing it, is to consider the LHS
\begin{equation}
    \begin{aligned}
        \text{LHS} &= P(X \in A) - P(Y \in A) \\
                & = P(X \neq Y, X\in A) - P(X \neq Y, Y\in A)\\
                & \leq P(X \neq Y) 
    \end{aligned}
\end{equation}

\section*{HW5}
\subsection*{Q3}
\(\lambda \) is the average 2. 
Therefore, the probability of scoring exactly 6 in one game is 
\[
   p =   e^{-\lambda} \frac{\lambda^6}{6!}
\] 
The problem ask for the probability that 'at least' one game scoring exactly 6 Field goals, which has the probability
\(1- (1-p)^{150}\) 
\subsection*{Q4}
Check the note for recitation on Google drive. 
\end{document}