\documentclass[12pt]{article}

\usepackage[english]{babel}

\usepackage[letterpaper,top=2cm,bottom=2cm,left=3cm,right=3cm,marginparwidth=1.75cm]{geometry}

% Useful packages
\usepackage{amsmath}
\usepackage{amsfonts}
\usepackage{dsfont}
\usepackage{graphicx}
\usepackage[colorlinks=true, allcolors=blue]{hyperref}
%\usepackage[legalpaper, landscape, margin=1in]{geometry}
\usepackage{geometry}
\geometry{
a4paper,
left=20mm,
right=20mm,
top=15mm,
}
\usepackage{setspace}
\setstretch{1.25}

\begin{document}
\section*{Quiz 1: Give yourself 50 minutes to solve 5 of the following 7 problems. Each problem weights 20 point scores}
\subsection*{Problem 1} For any two events $A$ and $B$ in $\Omega$, verify that $\mathbb{P}(A\cup B)=\mathbb{P}(A)+ \mathbb{P}(B)- \mathbb{P}(A\cap B)$.
For three events $A$, $B$ and $C$, prove that
\begin{equation*}
\begin{aligned}
\mathbb{P}(A_1\cup A_2\cup A_3) &= \mathbb{P}(A)+\mathbb{P}(B)+\mathbb{P}(C) \\&
-\mathbb{P}(A\cap B)-\mathbb{P}(A\cap C)-\mathbb{P}(B\cap C) \\&
+\mathbb{P}(A\cap B\cap C).
\end{aligned}
\end{equation*}
\textbf{Solution.}
\begin{enumerate}
   \item By the axiom of disjoint additivity for probability,
   \begin{equation*}
       \begin{aligned}
        \mathbb{P}(A_1\cup A_2) &=\mathbb{P}(A_1\cup(A_2\setminus A_1))=\mathbb{P}(A_1)+\mathbb{P}(A_2\setminus A_1) , \qquad \text{and}\\
        \mathbb{P}(A_2) &=\mathbb{P}((A_2\setminus A_1)\cup(A_2\cap A_1))=\mathbb{P}(A_2\setminus A_1)+\mathbb{P}(A_2\cap A_1).
       \end{aligned}
   \end{equation*}
   Therefore, $\mathbb{P}(A_1\cup A_2)=\mathbb{P}(A_1)+\mathbb{P}(A_2)-\mathbb{P}(A_2\cap A_1)$. 
    $\hspace{\fill}\square$

   \item With the above identity,
    \begin{equation*}
    \begin{aligned}
    \mathbb{P}(A_1\cup A_2\cup A_3) &= \mathbb{P}(A_1\cup A_2)+\mathbb{P}(A_3)-\mathbb{P}((A_1\cup A_2)\cap A_3) \\&
    =\mathbb{P}(A_1)+\mathbb{P}(A_2)-\mathbb{P}(A_1\cap A_2)+\mathbb{P}(A_3)-\mathbb{P}((A_1\cap A_2)\cup(A_1\cap A_3)) \\&
    =\mathbb{P}(A_1)+\mathbb{P}(A_2)-\mathbb{P}(A_1\cap A_2)+\mathbb{P}(A_3)
    \\&
    \hspace{5mm}-\mathbb{P}(A_1\cap A_2)-\mathbb{P}(A_1\cap A_3)+\mathbb{P}(A_1\cap A_2\cap A_3).
    \end{aligned}
    \end{equation*}
    $\hspace{\fill}\square$
\end{enumerate}

\subsection*{Problem 2} (Matching problem*) There are a deck of $n$ distinct cards and $n$ distinct boxes. Shuffle
the cards and placed them into the boxes (Only one card for each boxes), if the $i$-th card
is placed at the $i$-th box, we say that there is a match. What is the probability of no
match after a shuffling? Compute the limiting probability when $n\rightarrow\infty$.
\\
\textbf{Solution.} Let $A_i$ denotes the event that the $i$-th card is placed in the $i$-th box. Then the desired probability is
\begin{equation*}
\mathbb{P}(A_1^c\cap\cdots\cap A_n^c)=1-\mathbb{P}(A_1\cup\cdots\cup A_n).
\end{equation*}
Also, for any distinct indices $i_1,\cdots,i_k\in\{1,\cdots,n\}$, $1\leq k\leq n$,
\begin{equation*}
\mathbb{P}(A_{i_1}A_{i_2}\cdots A_{i_k})=\frac{(n-k)!}{n!}.
\end{equation*}
Then by the inclusion-exclusion formula, the desired probability is
\begin{equation*}
\mathbb{P}(A_1^c\cap\cdots\cap A_n^c)=1-\mathbb{P}(A_1\cup\cdots\cup A_n)=1-\sum_{k=1}^n\binom{n}{k}\frac{(n-k)!}{n!}=\sum_{k=0}^n(-1)^k\frac{1}{k!}.
\end{equation*}
And the limiting probability is
\begin{equation*}
\lim_{n\rightarrow\infty}\mathbb{P}(A_1^c\cap\cdots\cap A_n^c)=\lim_{n\rightarrow\infty}\sum_{k=0}^n\frac{(-1)^k}{k!}=e^{-1}.
\end{equation*}
$\hspace{\fill}\square$
\\ \\
\subsection*{Problem 3} (Matching problem, continuation**) What is the probability of exactly $m$ matches, $1\leq m\leq n$?
\\
\textbf{Solution.} By the answer of problem 2, we can know that when we put $m$ cards into $m$ boxes, the number of distributions such that there is no match is $m!\times\mathbb{P}(\text{no match})=m!\sum_{k=0}^m(-1)^k\frac{1}{k!}$. So the total number of distributions such that there has exactly $m$ matches is
\begin{equation*}
\#\{\text{ways to choose $m$ matches}\}\times \#\{\text{ways to place remaining $n-m$ cards with no match}\}.
\end{equation*}
Divided by the total possible outcomes $n!$, we get
\begin{equation*}
\mathbb{P}(\text{exactly $m$ matches})=\frac{\binom{n}{m}(n-m)!\sum_{k=0}^{n-m}(-1)^k\frac{1}{k!}}{n!}=\frac{1}{m!}\sum_{k=0}^{n-m}(-1)^k\frac{1}{k!}.
\end{equation*}
$\hspace{\fill}\square$
\\ \\
\subsection*{Problem 4} You have $n$ pairs of socks. If $2r$ socks was chosen randomly, what’s the probability of getting exactly $i$ pairs of socks?
\\
\textbf{Solution.} The total number of choose $2r$ socks from $n$ pairs is $\binom{2n}{2r}$. The number of the desired choices can be count as follows: First choose the $i$ pairs from $n$ pairs, hence $\binom{n}{i}$ choices. For the remaining $2r-2i$ choices, we can't choose any pair from the remaining $n-i$ pairs, so we can at most pick one sock from each remaining pairs, therefore $\binom{n-i}{2r-2i}\times 2^{2r-2i}$. Hence the probability of choose exactly $i$ pairs is $\frac{\binom{n}{i}\binom{n-i}{2r-2i} 2^{2r-2i}}{\binom{2n}{2r}}$.
$\hspace{\fill}\square$
\\ \\
\subsection*{Problem 5}
Roll a fair die $10$ times. Compute the probability that at least one number occurs
exactly $6$ times.
\\
\textbf{Solution.} Define $A_i$ as the event that the number $i$ occurs exactly $6$ times in ten rolls, $1\leq i\leq 6$. Then the desired probability is $\mathbb{P}(A_1\cup A_2\cup\cdots\cup A_6)$. Also, for each $1\leq i\leq 6$, $\mathbb{P}(A_i)=\binom{10}{6}(6-1)^{10-6}/6^{10}$, hence by the inclusion-exclusion formula,
\begin{equation*}
\begin{aligned}
\mathbb{P}(A_1\cup\cdots\cup A_6)&=\sum_{i=1}^6\mathbb{P}(A_i)-\sum_{i,j}\mathbb{P}(A_iA_j)+\sum_{i,j,k}\mathbb{P}(A_iA_jA_k)-\cdots \\&
=6\times\frac{\binom{10}{6}5^4}{6^{10}}=\frac{\binom{10}{6}5^4}{6^{9}}.
\end{aligned}
\end{equation*}
$\hspace{\fill}\square$
\subsection*{Problem 6} Given two integers $N$ and $K$. The task is to find the number of good permutations of the first $N$ natural numbers. A permutation is called good if there exist at least $N-K$ indices $i,\,1\leq i\leq N$, such that $i$ is at the $i$th position. What is the probability of getting good permutations if $N= 6,\, K=3$?
\\
\textbf{Solution.} This is an application of the previous matching problem. Let $A_i$ be the event of getting a permutation with exactly $i$ correct positions. Then the desired probability is
\begin{equation*}
\mathbb{P}(\text{at least $N-K$ correct positions})=\mathbb{P}\Big(\bigcup_{l=N-K}^NA_{l}\Big)=\sum_{l=N-K}^N\mathbb{P}(A_l),
\end{equation*}
since the events are disjoint. Also, by Problem 4, 
\begin{equation*}
\mathbb{P}(A_{l})=\frac{1}{l!}\sum_{k=0}^{N-l}(-1)^k\frac{1}{k!},
\end{equation*}
Thus $\mathbb{P}(A_3)=\frac{1}{18}$, $\mathbb{P}(A_4)=\frac{1}{48}$, $\mathbb{P}(A_5)=0$, $\mathbb{P}(A_6)=\frac{1}{720}$. So the answer is $56/720$.
$\hspace{\fill}\square$
\hspace{1 \textwidth}
\\ \textbf{Another solution.} 
A cheap way of calculating the answer is to enumerate the number of cases for $A_3, A_4, A_5, A_6$, and then divide the total number of cases by $|\Omega| = 6! $ to get the answer. 
\[
    |A_l| = \underbrace{\begin{pmatrix}
         6 \\
         l \\
    \end{pmatrix} }_{\# \text{l number at good position} }\times (\# \text{derangement for }  (1,2,\dots, n-l))
\]
The length of derangement in this problem can be counted directly.
\\ \\
\subsection*{Problem 7} In a school, three-quarters of students are involved in sports, half are involved in cultural activities, and one-eighth are involved in neither. Calculate the probability that a student is involved in\\
(a) both sports and cultural activities\\
(b) cultural activities but not sports.\\
\\
\textbf{Solution.} 
\begin{itemize}
   \item[(a)] By inclusion-exclusion principle, we have
\begin{equation*}
\mathbb{P}(\text{both sports and cultural activities})=\frac{3}{4}+\frac{1}{2}-\Big(1-\frac{1}{8}\Big)=\frac{3}{8}.
\end{equation*}
    \item[(b)] \begin{equation*}
\mathbb{P}(\text{cultural activities but no sport})=\frac{1}{2}-\frac{3}{8}=\frac{1}{8}.
\end{equation*}
\end{itemize}
 

$\hspace{\fill}\square$


\end{document}