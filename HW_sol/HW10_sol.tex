\documentclass[12pt]{article}

\usepackage[english]{babel}

\usepackage[letterpaper,top=2cm,bottom=2cm,left=3cm,right=3cm,marginparwidth=1.75cm]{geometry}

% Useful packages
\usepackage{amsmath}
\usepackage{amsfonts}
\usepackage{dsfont}
\usepackage{graphicx}
\usepackage[colorlinks=true, allcolors=blue]{hyperref}
%\usepackage[legalpaper, landscape, margin=1in]{geometry}
\usepackage{geometry}
\geometry{
a4paper,
left=25mm,
right=25mm,
top=15mm,
}
\usepackage{setspace}
\setstretch{1.25}

\begin{document}
\newcommand*{\horzbar}{\rule[.5ex]{2.5ex}{0.5pt}}

\section*{Partial solution to HW 10}
\subsection*{Heads up}
Q1 to Q4 will be in the range of the finals. Q5 to Q7 are optional (If you want to know what is Lyapunov CLT) 

For notation convenience, in the following content we assume \(g \sim N(0,1)\) and \(P(g < t) = \Phi(t) = \int_{-\infty }^{t} \frac{1}{\sqrt{2 \pi }} e^{- \frac{t^2}{2}} dt \).  
Also, since the p.d.f. of \(g\) is an even function, we have \(\Phi(-t) = 1-\Phi(t)\).  
\subsection*{Q1}
Since we have i.i.d. \(X_n \sim \text{Ber}(p) \) for all \(n\), \(E[X_n] =p\) and \(\text{Var} (X_{n} ) = p(1-p)\) less than infinity, so we can use CLT. 
The thing to find is  
\[
    \lim\limits_{n \to \infty} P(\left\vert \frac{S_{n}}{n} -p \right\vert > t_{n}) 
\] 
with the condition that \(\lim\limits_{n \to \infty} \sqrt{n} t_{n} \to t \in [0,\infty )\). 
To apply CLT, we should put the problem in a CLT form. That is, 
\[
    P(\left\vert \frac{S_{n}}{n} -p \right\vert > t_{n}) = P(\left\vert \frac{S_{n} - np}{\sqrt{n} } \right\vert > \sqrt{n} t_{n}) 
\]
It will not hurt a lot if we just assume \(\sqrt{n} t_{n}\) is \(t\). In that case, we can apply CLT to get 
\[
    \lim\limits_{n \to \infty} P(\left\vert \frac{S_{n} - np}{\sqrt{n} } \right\vert > t) \underbrace{=}_{CLT} P(g<-t) + P(g > t) = \Phi(-t) + (1-\Phi(t)) = 2\Phi(-t) 
\]
\hspace{\textwidth}\(\square\) 

\textbf{Note:} For those interested, the following are the procedure to follow if you don't cut corner.  

To deal with the changing number \(\sqrt{n} t_{n} \), we may use limsup and liminf to show the limit exists. 
Since  
\[
    \lim\limits_{n \to \infty} P(\left\vert \frac{S_{n}}{n} -p \right\vert > t_{n}) 
\]
For an arbitrary \(\epsilon >0\), there exist \(N_{\epsilon }\in \mathbb{N}\) such that \(\left\vert \sqrt{n} t_{n} -t \right\vert < \epsilon \). Hence, for \(n > N\)  
\[
 \underbrace{P(\left\vert \frac{S_{n} - np}{\sqrt{n} } \right\vert > t + \epsilon )}_{T_{t+\epsilon }} \leq \underbrace{ P(\left\vert \frac{S_{n} - np}{\sqrt{n} } \right\vert > \sqrt{n} t_{n}) }_{T_n} \leq \underbrace{P(\left\vert \frac{S_{n} - np}{\sqrt{n} } \right\vert > t-\epsilon )}_{T_{t-\epsilon }} 
\]
Hence, we will have  
\[
   2\Phi(-t -\epsilon )     \underbrace{=}_{CLT} \lim\limits_{n \to \infty} T_{t + \epsilon} \leq  \liminf\limits_{n\rightarrow \infty } T_{n}   \leq \limsup\limits_{n\rightarrow \infty } T_{n}  \leq \lim\limits_{n \to \infty} T_{t-\epsilon} \underbrace{=}_{\text{CLT}} 2\Phi(-t + \epsilon )  
\]
Since \(\epsilon\) is arbitrary, we may take \(\epsilon \to 0+\). 
Then, limsup and liminf are the same, which means the limit of \(T_n\) exists and equals \(2\Phi(-t)\) 

\begin{equation*}
    \begin{aligned}
        a
    \end{aligned}
\end{equation*} 
\subsection*{Q2}
Recall that the stability of Poission r.v. said that for two independent \(X_1 \sim \text{Pois} (\lambda_1)\) and \(X_2 \sim \text{Pois} (\lambda_2)\), then \(X_1 + X_2 \sim \text{Pois}(\lambda_1 + \lambda_2) \). 
Another thing, we'll take it as a fact that \(E[X_1] = \text{Var}(X_1) = \lambda_1  \) 
\\
In our case, we want to make a summation term so it looks like what we have for CLT. Due to stability of Poission we may view \(X\) as the sum of i.i.d. Poisson r.v. 
,i.e. \(X = \sum_{i=1}^{n} X_i\), where \(X_i \sim \text{Pois}(\frac{\lambda}{n}) \) 
\\Then, we can reorganize the given term as 
\[
    P(\left\vert X - \lambda  \right\vert \geq a \sqrt{\lambda }  ) = P(\frac{\left\vert X - n(\frac{\lambda}{n}) \right\vert}{\sqrt{n}\sqrt{\frac{\lambda}{n}}  }  \geq a)
\]
Hence, by CLT we have the limit being
\[
    \lim\limits_{n \to \infty}  P(\left\vert X - \lambda  \right\vert \geq a \sqrt{\lambda }  ) \underbrace{=}_{CLT} P(g \leq  -a) + P(g \geq a) = \Phi(-a) + (1-\Phi(a)) = 2\Phi(-a), a > 0 
\]
\hspace{\textwidth}\(\square\) 
\subsection*{Q3}
First step, turn the words into useful things. We may answer the problem by estimating how likely will this situation happen. 
Here we assume that \(96\) people is a lot ($=\_=$). 
\\ 
What's next? How do we know the standard deviation? \\
To answer this, we have to make assumption about the selection process. In this case it is reasonable to assume that for each selection \(X_i\), it's either from minority group with probability \(p = 0.24\) or not with probability \(1-p\). 
Then, the sum \(S_n = \sum_{i=1}^{96}X_i \) will be the number of people from minority groups. Note that \(E[X_i] = p\), \(E[S_{96} ] \approx 23\),  and \(\text{Var}(X_i) = p(1-p) \approx 0.182\).  Hence, we are calculating 
\[
   P(S_{96} \leq 13) = P(\frac{S_{96} - 23}{\sqrt{96 \cdot 0.18} } \leq \frac{-10}{\sqrt{96 \cdot 0.18} }) \underbrace{\approx}_{CLT} \Phi(-2.41) 
\]   
Checking the value at \href{https://www.z-table.com/}{z-table}, it's around \(0.8\%\), so really unlikely. 
For more about jury selection process, check out \href{https://youtu.be/cPRK_ABldIk}{Vox video}.  
\subsection*{Q4}

\end{document}